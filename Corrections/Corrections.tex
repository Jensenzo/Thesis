\clearpage
\thispagestyle{empty}
Jens Christensen

List of Corrections

\textbf{Overall Corrections}

\begin{enumerate}
  \item All notes in thesis corrected.
  \item Labels added to figures missing it and colours changed in Figures~\ref{fig:SeparationWaveformEx320.pdf} (pp. \pageref{fig:SeparationWaveformEx320.pdf}), \ref{fig:SeparationWaveformEx640.pdf} (pp. \pageref{fig:SeparationWaveformEx640.pdf}), \ref{fig:SeparationWaveformExBig320.pdf} (pp. \pageref{fig:SeparationWaveformExBig320.pdf}) and \ref{fig:SeparationWaveformExBig640.pdf} (pp. \pageref{fig:SeparationWaveformExBig640.pdf}).
  \item The abstract remains an accurate representation of the work undertaken and the conclusions drawn throughout the thesis.
  \item Done. See beginning of thesis.
  \item On page~\pageref{corrections:motherWavelet} the mother wavelet basis used throughout this work is presented and is the Daubechies 8 (db8) wavelet \cite{Daubechies1992}.
\end{enumerate}

\textbf{Abstract}
\begin{enumerate}
  \item The abstract remains an accurate representation of the work undertaken and the conclusions drawn throughout the thesis.
\end{enumerate}

\textbf{Chapter 2}
\begin{enumerate}
\item \gls{mfcc} has been added in the literature review (pp. \pageref{corrections:mfcc1}) as a major audio feature in speech feature extraction and as proposed future work (pp. \pageref{corrections:mfcc2}).
    \item On pp. \pageref{corrections:viterbi} the Viterbi algorithm is introduced as a recursive optimal solution to the problem of estimating the state sequence of a discrete-time finite-state Markov process. A sentence has been added to clarify how the Viterbi may be used for a classification application given certain circumstances.
    \item On pp. \pageref{fig:LitRev_DetectCompare} and \pageref{fig:LitRev_DetectCompare2} references have been added to indicate the inspiration for the particular method's use in the comparison.
    \item The patent mentioned in the list of corrections was already referenced (a long with other occurrences of the same IP) in reference \cite{Seltzer2011Patent} and does not appear to be related to patent \cite{US8233353}. A new section that presents and discusses the state of the art of keyboard stroke removal algorithms has been included starting on page \pageref{corrections:methods} including the recent Sugiyama, A paper mentioned in the correction notes.
    \end{enumerate}
    
Chapter 3
\begin{enumerate}
\item The number of taps included in the training and testing sets $D^1$ and $D^2$ were already included on page \pageref{corrections:DSNcount}. The sets comprised approximately 70 taps per set.
\item A paragraph giving further details about the generation of the training and testing sets $D^1$ and $D^2$ have now been included on page \pageref{corrections:DSNmethod}.
\item Cut-off frequency used was 1102.5 Hz (pp. \pageref{corrections:cut-off}). Spectrogram example of performance on Figure~\ref{fig:filterCompareSpectrogram} pp. \pageref{fig:filterCompareSpectrogram}.
\item Data presented over various tapping position in Table~\ref{tab:APRresultsPerChan} on page \pageref{tab:APRresultsPerChan}. These results have also been included in the discussion and comments surrounding the table in the text.
\item LOO discussion and general testing methodology added. Specifically on page \pageref{corrections:LOO}.
\item Added a small paragraph to the discussion in chapter 3 describing the possibility of \gls{ml} and \gls{map} in the context of audio classification had an appropriate feature presented itself (pp. \pageref{corrections:ML}.
\item A section on the \gls{gmm} as a possible classifier for the \gls{apr} system is included on section \ref{corrections:GMM} page \pageref{corrections:GMM}.
\end{enumerate}

\textbf{Chapter 4}
\begin{enumerate}
\item Have added a short paragraph explaining why a wooden board may in fact be a valuable proxy for an actual screen for this application (pp. \label{corrections:wooden}.
\item Added. See Figure~\ref{fig:MultiAPRsystem.pdf} page~\pageref{fig:MultiAPRsystem.pdf}.
\item Per spot resolution of results shown in Table~\ref{tab:multiAPRresultsPerChan} on page~\pageref{tab:multiAPRresultsPerChan}.
\end{enumerate}

\textbf{Chapter 5}
\begin{enumerate}
\item More details added on page~\pageref{corrections:library}.
\item More information and description added on page~\pageref{corrections:further_quantify}. Apparent discrepancy discussed in two paragraphs starting on page~\pageref{corrections:discuss_discrep}.
\end{enumerate}

\textbf{Chapter 6}
\begin{enumerate}
\item Colours used in figures did no print well. Blue colour replaced with bright green to increase contrast throughout Chapter 6.
\item Changed on page~\pageref{eq:RestBasicModel}.
\item \emph{Bump} placement indicated in caption of figure~\ref{fig:TonalArtefactSpectrumExample.png} page~\pageref{fig:TonalArtefactSpectrumExample.png}.
\item Additional details about the subjects and details surrounding the informal listening test has been supplied on page~\pageref{corrections:subjects}.
\item Table~\ref{table:ListenerTestFiles} included giving details about test files used in the listening test on page.~\pageref{table:ListenerTestFiles}.
\item Two paragraphs included from page~\pageref{corrections:conclusionSubjective} discusses the value of further subjective testing.
\end{enumerate}