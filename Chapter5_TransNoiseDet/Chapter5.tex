\chapter{Transient Noise Detection}\label{ch:TransientNoiseDetection}

\ifpdf
    \graphicspath{{Chapter5_TransNoiseDet/Chapter5Figs/PNG/}{Chapter5_TransNoiseDet/Chapter5Figs/PDF/}{Chapter5_TransNoiseDet/Chapter5Figs/}}
\else
    \graphicspath{{Chapter5_TransNoiseDet/Chapter5Figs/EPS/}{Chapter5_TransNoiseDet/Chapter5Figs/}}
\fi

The rapid increase in availability of high speed internet connections has made personal computers a popular basis for teleconferencing applications. While embedded microphones, loudspeakers and webcams in laptop computers have made setting up conference calls very easy, it has also brought with it some specific noise difficulties such as feedback, fan noise and button clicking noise. The latter has been a particularly persistent problem and is generally due to the mechanical impulses caused by keystrokes. Particularly on laptop computers this can be a significant nuisance due to the mechanical connection between microphone, within in the laptop case, and the keyboard. The noise pulses produced can vary greatly with factors such as keystroke speed and length, microphone placement and response, laptop frame or base, keyboard or trackpad type and even surface on which the computer is placed.

As noted in the literature review, chapter~\ref{sec:LitRev_Detection}, a range of approaches has been taken to detect impulsive noise in speech and audio. In general the approaches taken can be divided into two categories. The ones that focus on corruptions caused by mechanical defects in the medium or errors in the communication channel and 

\section{Separation pre-filtering}
\section{Noise burst model}
\section{AR filtering approach}

% ------------------------------------------------------------------------


%%% Local Variables:
%%% mode: latex
%%% TeX-master: "../thesis"
%%% End:
