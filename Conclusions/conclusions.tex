\def\baselinestretch{1}
\chapter{Conclusions}\label{ch:Conclusions}
\ifpdf
    \graphicspath{{Conclusions/ConclusionsFigs/PNG/}{Conclusions/ConclusionsFigs/PDF/}{Conclusions/ConclusionsFigs/}}
\else
    \graphicspath{{Conclusions/ConclusionsFigs/EPS/}{Conclusions/ConclusionsFigs/}}
\fi

\def\baselinestretch{1.6}

% Possible to accurately estimate the origin of touch interactions. Dependent on scenario... background noise... temperature shifts... sampling rate... device size... microphone placement and type.

This thesis has explored two aspects of impact induced pulses in real-time audio streams. Firstly a functional view of pulses that aimed to localise impact sites of touch events solely based on the resulting pulses in the audio stream for primarily single-channel applications with a multi-channel extension. This application was referred to as acoustic pulse recognition (APR) and at its core relied on being able to distinguish some minute variations in transient pulses while accepting other variations. Factors such as background noise, tapping style, strength, apparatus and even environmental temperature was shown to have a significant effect on the pulse waveform, an effect that the algorithm would need to be able to deal with. A range of waveform matching algorithms were developed, tested and some applications were hypothesised and tested.
Secondly, the pulses were considered noise and a real-time single-channel detection and restoration system was constructed for a telecommunication application. Various detection algorithms were proposed and tested, but for the unsupervised generally applicable case some key observations about the noise pulses were the guiding principles of work: noise pulses were typically broad-spectrum events with rapid changing energy levels which formed the necessary contrast to speech signals. After successful detection the restoration was considered in two independent steps based on the two outputs of the pre-processing algorithm used for the application. A set of largely clean tonal atoms and the residual. Here the residual was considered as the primary focus for the array of restoration approaches trialled, but heuristic approaches were explored for the restoration of noise polluted tonal components as well.

This concluding chapter sums up relevant results from throughout the thesis and conclusions are presented based on the aims set out in the introduction. Finally selected suggestions for future work are presented.

\section{Summary of results}
%Results also showed that pulses could successfully be modelled as linear scalings of a combination of components, yielding further improved performance.
Results presented in chapters~\ref{ch:APR} and~\ref{ch:MultichannelAPR} indicate a false classification rate of less, and in some cases far less, than 10 \% in the ideal environment. Both the single- and multi-channel systems were negatively affected by coloured background noise, but the multi-channel system managed to retain a near perfect classification rate. Results also showed that pulses could successfully be modelled as a linear scaling of components $I$, yielding further performance increases while computational complexity scaled linearly with $L$ number of scales introduced. $L=20$ scales were found to be sufficient, above which no performance increase was detected. The PCA factorisation approach achieved a high degree of separability between components and $I=3$ was found to be an adequate number of components in relation to the component number's influence on performance, and template lengths of about 120 samples equally seemed to indicate the point of diminishing returns for the system.

Results from the detection algorithm were presented in chapter~\ref{ch:TransientNoiseDetection} where two wavelet based detectors were presented both showing similar or better performance characteristics than comparable methods from the literature with the added advantage of higher temporal resolution. A pre-processing stage was presented that took advantage of the difference in tonality which was shown to effectively separate speech and transient noise both audibly and visibly. The pre-processing stage also provided the basis for the two-level approach to restoration proposed in chapter~\ref{ch:TransientNoiseRestoration} where the tonal and residual components were treated separately, if at all necessary in the case of the tonal components. A variety of objective perceptual methods showed general modest improvements of 0.2 MOS and 0.4 for PEAQ for the selected sets which in subjective tests were not favored by listeners. Overall subjective tests also revealed a slight preference to the restored samples with huge variability in preference between different sets tested.

\section{Evaluation of results}
Work presented on the functional aspects of impact induced audio pulses have clearly shown that it is possible to implement a simple touch based user interface system using APR with only one active channel. Allowing for more computationally intensive methods (K-PCA) increased performance significantly in clean as well as noisy environments. Even the simpler univariate methods outperformed the random chance baseline by a margin. The more advanced PCA based method proved more resilient to the common variability in pulse waveforms associated with a variety of factors. This was only the case if these could be incorporated into the model during training while the simpler methods failed in this regard. A multi-channel extension/generalisation of APR further increased classification performance to near perfect levels with the testing data used in the testing scenarios.

The concept of a single-channel APR system has been proposed and clearly shown to be a possible and feasible solution to the problem of touch interactive user interfaces. Some key difficulties have been explored and to a large part tackled, such as variability in tapping styles and strength.
%possibly more here though?

A noise pulse detection and restoration system for the telecommunication application was designed and realised. Tests show that a real-time unsupervised single-channel keyboard typing suppression system is viable both subjectively and objectively. The detection system performs similarly or better than state of the art alternatives using more limited approaches. The proposed method requires minimal look-ahead time and provides a much higher temporal resolution than alternative methods due to the application of the wavelet basis. The separation stage and the two-level restoration enables a scalable restoration approach with tonal restoration providing an additional line of restoration for keyboard pulses showing strong tonal components. The high temporal resolution would in particular be of utility in scenarios with corruption of far shorter extent such as clicks or transmission errors.

\section{Suggestions for future research}
In this section a selection of the future research suggestions presented throughout this thesis has been collected. No new suggestions are presented here but in some cases proposals have been combined across chapters and topics.

%***************** Chap 3
% Better PCA component choices
% Improve and extend Sparse Training to PCA
\subsection{Sparse Training applied to PCA and common PCA components}
The preliminary study of the sparse training extension showed that this approach is possible with even the simplest interpolation approaches. A study into the extent to which individual components in the PCA approach vary over a surface could lead to a straight forward PCA implementation of the sparse training approach. While this first approach assumes that equivalent components are identifiable at neighbouring active spots, it might be easier to achieve this goal by drawing the random components from an ensemble of the entire training set data (i.e. from all models) and then identifying their contribution in each model. This would have the additional advantage of compressing the training data needed and possibly also provide a different approach to filtering out non-essential components for classification.

%***************** Chap 4
%correlate noise
\subsection{Correlated noise}
While the assumption of uncorrelated noise in the multi-channel APR system holds for some cases, such as faint localised and own-noise, more significant noise, such as the background music used or speech, would be correlated across channels. Extending the proposed model to incorporate correlated noise might further increase the robustness of the multi-channel APR system to false detections or false classifications.

%***************** Chap 5
\subsection{Multi-channel detection and restoration}
Based on the same background assumption that led to the development of the multi-channel APR system in chapter~\ref{ch:MultichannelAPR}, it is possible to extend the detection process to multiple channels. While not necessarily providing the same information through ToF it is still possible to improve performance through additional correlated sources as clearly shown from the results in chapter~\ref{ch:MultichannelAPR} if only with nothing more than simple majority voting (median/mode processing).

With additional channels, restoration performance could feasibly also be increased in a variety of ways depending on the implementation of the additional channels. Assuming that additional channels could be considered reference sources for either the pulse or that the speech could provide valuable cues in relation to tonal components of relevance and noise-related components. A simple restoration approach could simply involve subtraction, possibly given some alteration, of the channels, which could, in the case of a good speech reference, form the basis of an alternative to the pre processing separation stage.

%***************** Chap 6
\subsection{Modified restoration algorithms}
Based on the work presented in \cite{Esquef2006} sequential implementations of the energy adjusted LSAR algorithms could form the basis of better sounding restoration algorithms for the residual component of the signal. A modified pre-processing separation stage could include more information about what the detected tonal components are likely to look like over time. Peak tracking or sinusoidal tracking schemes\cite{McAulay1986} could potentially dismiss the spurious tonal noise components seen with some keyboards while aiding in tonal restoration by providing a statistical basis for reconstructing lost tonal information during a corruption.

\subsection{Correct detection verification}
Post restoration comparison of the restored and original signal should reveal a significant decrease in maximum MSE or some other objective metric. This in turn could provide a useful way of confirming that a detection was in fact caused by a noise pulse and therefore justified.















%http://www.devstud.org.uk/downloads/4be165997d2ae_Writing_the_Conclusion_Chapter,_the_Good,_the_Bad_and_the_Missing,_Joe_Assan%5B1%5D.pdf


%The conclusion chapter should have a definitive introduction which draws the attention of the
%reader to the thesis statement upon which the research was conducted. The introduction
%should restate the research question that the study set out to answer and clearly justify the
%necessity of such a course. There is also the need to establish the context, background
%and/or importance of the topic. Conversely, this section must indicate a problem, controversy
%or a gap in the field of study. In doing this it is proper that the research questions are outlined
%and the key objectives of the study.
%
%Secondly, the introduction of the conclusion, just like those of the discussion chapters should
%provide a map of how the chapter has been structured. It should therefore provide a pictorial
%sequence of the issues to be discussed and how the section will end. This allows the
%examiner the opportunity to know what to expect and strong grounding for the research
%coverage.


%%% ----------------------------------------------------------------------

% ------------------------------------------------------------------------

%%% Local Variables:
%%% mode: latex
%%% TeX-master: "../thesis"
%%% End:
