0\def\baselinestretch{1}
\chapter{My Conclusions ...}\label{ch:Conclusions}
\ifpdf
    \graphicspath{{Conclusions/ConclusionsFigs/PNG/}{Conclusions/ConclusionsFigs/PDF/}{Conclusions/ConclusionsFigs/}}
\else
    \graphicspath{{Conclusions/ConclusionsFigs/EPS/}{Conclusions/ConclusionsFigs/}}
\fi

\def\baselinestretch{1.66}

% Possible to accurately estimate the origin of touch interactions. Dependent on scenario... background noise... temperature shifts... sampling rate... device size... microphone placement and type.


\section{Summary of results}

\section{Evaluation of results}



\section{Suggestions for future research}

%http://www.devstud.org.uk/downloads/4be165997d2ae_Writing_the_Conclusion_Chapter,_the_Good,_the_Bad_and_the_Missing,_Joe_Assan%5B1%5D.pdf


%The conclusion chapter should have a definitive introduction which draws the attention of the
%reader to the thesis statement upon which the research was conducted. The introduction
%should restate the research question that the study set out to answer and clearly justify the
%necessity of such a course. There is also the need to establish the context, background
%and/or importance of the topic. Conversely, this section must indicate a problem, controversy
%or a gap in the field of study. In doing this it is proper that the research questions are outlined
%and the key objectives of the study.
%
%Secondly, the introduction of the conclusion, just like those of the discussion chapters should
%provide a map of how the chapter has been structured. It should therefore provide a pictorial
%sequence of the issues to be discussed and how the section will end. This allows the
%examiner the opportunity to know what to expect and strong grounding for the research
%coverage.


%%% ----------------------------------------------------------------------

% ------------------------------------------------------------------------

%%% Local Variables:
%%% mode: latex
%%% TeX-master: "../thesis"
%%% End:
