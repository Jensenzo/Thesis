%%% Thesis Introduction --------------------------------------------------
\chapter{Introduction}
\ifpdf
    \graphicspath{{Introduction/IntroductionFigs/PNG/}{Introduction/IntroductionFigs/PDF/}{Introduction/IntroductionFigs/}}
\else
    \graphicspath{{Introduction/IntroductionFigs/EPS/}{Introduction/IntroductionFigs/}}
\fi

\section{The need for research}

\section{Aims and objectives}

\section{Scope of the thesis and contributions to knowledge}

\section{About this thesis}

\section{Terms used}
%pulse/impulse
%noise
%Transient noise used when focus on noise.
Throughout the literature the terms \emph{pulse}\cite{Esquef2002a}\cite{Esquef2003a} and \emph{impulse}\cite{Czyzewski1995}\cite{Kauppinen2002a}\cite{Chen2000} has been used interchangeably to refer to a short burst of sound. Generally the use of the term \emph{click}\cite{Czyzewski1995}\cite{Esquef2002}\cite{Godsill1998book} has been associated with physical defects in a recording medium or other extremely short term corruptions. Henceforth the term \emph{pulse} will be used to describe the specific audio signals of interest throughout this work, for its alignment with the already established term Acoustic Pulse Recognition (APR)\cite{TouchSystems2006}. The term \emph{noise} or \emph{transient noise} will only be used to refer to specifically unwanted aspects of the signal and will as such mainly be used in chapters~\ref{ch:TransientNoiseDetection} and \ref{ch:TransientNoiseRestoration}

%Keyboard stroke pulse (sometimes pulse sequence) vs. single pulse
In chapters~\ref{ch:TransientNoiseDetection} and \ref{ch:TransientNoiseRestoration} the term \emph{keyboard stroke} or \emph{key stroke} will be used to specifically refer the sequence of disturbances that arise from typing action on a keyboard. These sequences will in many cases encompass a series \emph{pulses}.


%%% ----------------------------------------------------------------------


%%% Local Variables:
%%% mode: latex
%%% TeX-master: "../thesis"
%%% End:
